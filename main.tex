% -*- coding: utf-8 -*-
%%
%%
%%
%%
%%
%%
%%  本模板可以使用以下两种方式编译:
%%
%%     1. PDFLaTeX
%%
%%     2. XeLaTeX [推荐]
%%
%%  注意:
%%    1. 在改变编译方式前应先删除 *.toc 和 *.aux 文件,
%%       因为不同编译方式产生的辅助文件格式可能并不相同。
%%
%%
\documentclass[12pt,openany]{book}
          
\usepackage{ifxetex}
\ifxetex
  \usepackage[bookmarksnumbered]{hyperref}
\else
  \usepackage[unicode,bookmarksnumbered]{hyperref}
\fi

%\usepackage[emptydoublepage]{NKThesis}   % 中文
\usepackage[emptydoublepage,English]{NKThesis} % 英文

%   根据需要选择 biblatex 宏包选项.
\usepackage[backend = bibtex8, defernumbers = true,  sorting=none,  style = nkthesis]{biblatex}
\hypersetup{colorlinks=true,
            pdfborder=0 0 1,
            citecolor=black,
            linkcolor=black}
\usepackage{tikz}
\usepackage{amsmath}
\usepackage{amssymb}
\usepackage{multirow}
\usepackage{array}
\usepackage{arydshln}
%\usepackage{cchead}
\usepackage{enumitem}

\usepackage{bm}
%\usepackage{mathpazo}
\usetikzlibrary{arrows}
\usepackage{flowchart}
\usetikzlibrary{shapes.geometric}
\usepackage{listings}
\usepackage{xcolor}
%\usepackage{showkeys} %显示出\ref里的label代号,记得注释掉

\newcommand*{\QEDB}{\hfill\ensuremath{\square}}

\makeatletter
\renewcommand*\env@matrix[1][\arraystretch]{%
  \edef\arraystretch{#1}%
  \hskip -\arraycolsep
  \let\@ifnextchar\new@ifnextchar
  \array{*\c@MaxMatrixCols c}}
\makeatother
\newcommand{\wuhao}{\fontsize{10.5pt}{16pt}\selectfont}

\addbibresource{nkthesis.bib}
\DeclareBibliographyCategory{cited}
\AtEveryCitekey{\addtocategory{cited}{\thefield{entrykey}}}

\includeonly{
./tex/abstract,
./tex/ch1,
./tex/ch2,
./tex/ch3,
./tex/ch4,
./tex/ch5,
./tex/ch6,
./tex/ch7,
./tex/references,
./tex/acknowledgements,
%./tex/appendix,
./tex/resume
}


\newtheorem{thm}{Theorem}[chapter]
\newtheorem{Lemma}{Lemma}[chapter]
\newtheorem{Corollary}{Corollary}[chapter]
\newtheorem{Proposition}{Proposition}[chapter]
\newtheorem{Definition}{Definition}[chapter]
\newtheorem{example}{Example}[chapter]
\newtheorem{algo}{Algorithm}[chapter]
%\renewcommand{arraystrech}{1.5}
%\renewcommand\floatpagefraction{.9}
%\renewcommand\topfraction{.9}
%\renewcommand\bottomfraction{.9}
%\renewcommand\textfraction{.1}
\begin{document}

%  设置基本信息
%  注意:  逗号`,'是项目分隔符. 如果某一项的值出现逗号, 应放在花括号内, 如 {,}
%
\NKTsetup{%
  论文题目(中文) = 中文标题,
  副标题         = ,
  论文题目(英文) = 英文标题,
  论文作者       =  ,
  学号           =  ,
  指导教师       =  ,
  申请学位       = 理学硕士,
  培养单位       = ,
  学科专业       =  ,
  研究方向       =  ,
  中图分类号     = ,
  UDC            = ,
  学校代码       = 10055,
  密级           = 公开,
                   % 公开 | 限制 | 秘密 | 机密, 若为公开, 不填以下三项
  保密期限       = ,
  审批表编号     = ,
  批准日期       = ,
  论文完成时间   = 二〇一九年五月,
  答辩日期       = 2019/5/20,
  论文类别       = 学历硕士,
                   % 博士 | 学历硕士 | 硕士专业学位 | 高校教师 | 同等学力硕士
  院/系/所       = ,
  专业           =  ,
  联系电话       =  ,
  Email          = ,
  通讯地址(邮编) = 天津市卫津路94号南开大学统计研究院(300071),
  备注           = }


% -*- coding: utf-8 -*-


\begin{zhaiyao}



\end{zhaiyao}
\vspace{1cm}
\noindent
\begin{guanjianci}

\end{guanjianci}


\begin{abstract}



\end{abstract}
\vspace{1cm}
\noindent
\begin{keywords}

\end{keywords}

\tableofcontents
\include{./tex/ch1}
\include{./tex/ch2}
\include{./tex/ch3}
\include{./tex/ch4}
\include{./tex/ch5}
\include{./tex/ch6}
\include{./tex/ch7}
% -*- coding: utf-8 -*-

\renewcommand{\bibname}{Bibliography}

\begin{thebibliography}{EE}\wuhao 

\bibitem {} 

\end{thebibliography}




% -*- coding: utf-8 -*-

%\makeschapterhead{致谢}
\chapter*{Acknowledgements}


\bigskip

\begin{flushright}
XXX \\
Nankai University, Dec 2018
\end{flushright}


%\include{./tex/appendix}
% -*- coding: utf-8 -*-


\chapter*{Resume}

\noindent
{\bf Personal Information}
\begin{itemize}\wuhao
  \item [] Name: 
  \item [] Gender: 
  \item [] Address: 
  \item [] E-mail:
\end{itemize}

\noindent
{\bf Education}
\begin{itemize}\wuhao
\item [] 
\item [] \ \ \ \ \ \ \ \ \ \ \ \ \ \ \ \ \ \ \ \ \ \ \ \ \ \ \ \ \ \ \ 
\item [] 
\item [] \ \ \ \ \ \ \ \ \ \ \ \ \ \ \ \ \ \ \ \ \ \ \ \ \ \ \ \ \ \ \ \end{itemize}


\end{document}
